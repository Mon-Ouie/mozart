\section{The Reduction}

We define the set of inequations which are syntactically supported by
a constraint $\phi$, written $\phi \vdash x \leq y$, by the following
rules:
\[\frac{}{\phi \vdash x \leq x}\]
\[\frac{\phi \vdash x \leq y \qquad \phi \vdash y \leq z}{\phi \vdash
  x \leq z}
\]
\[\frac{f(x_1,\ldots,x_n)\leq x \in \phi
  \qquad \phi \vdash x \leq y
  \qquad y \leq g(y_1,\ldots,y_m) \in \phi}{\phi \vdash x_i \leq
  y_i}
\]

\begin{lemma}
  If $\phi\vdash x\leq y$, then $x\leq y$ holds in any solution of
  $\phi$.
\end{lemma}
\begin{proof}
Obvious.
\end{proof}

\subsection{OLDSTUFF}

We want to prove the following theorem:

\begin{satz}
\label{uppersemilatice}
Let the signature $\Sigma$ be an upper semi-lattice with a smallest
element. Satisfiability of non-structural subtyping constraints in the
structure of possibly infinte trees can be checked in cubic time.
\end{satz}

Let $\bot$ denote the smallest element of $\Sigma$. $\bot$ may be a
constant, or a function symbol with a non-zero arity.


We say that a constraint $\phi$ \emph{clashes} if it contains
constraints $f(x_1,\ldots,x_n) \leq x$ and $y \leq g(y_1,\ldots,y_m)$
with $f \not\leq g$, and if $\phi \vdash x \leq y$.

\begin{lemma}
  The constraint $\phi$ is satisfiable in the structure of possibly
  infinite trees if and only if it does not clash.
\end{lemma}

\begin{proof}
  Obviously, if $\phi$ clashes then it does not have a solution.
  
  Let $\phi$ be clash-free. We define the set of \emph{derived path
    constraints} by the following rules:

\[\ldots \]

\begin{proof}

\end{proof}

For any variable $z$ and path $\pi$, we define 
\[
L^z_\pi  := \{f \in \Sigma \mid \phi \vdash z \leq \pi(f) \}
\]
This set is finite even if the signature is infinite, since it can
only contain symbols that occur syntactically in the constraint.
Hence, any set $L^z_\phi$ has a supremum (which is $\bot$ in case the
set is empty).  We now define the valuation $\alpha$ by
\begin{itemize}
\item $ \alpha(z)(\epsilon) := \sup L^z_\epsilon$
\item For any path $\pi$ such that $\alpha(z)(\pi)=f$ and $1 \leq i
  \leq arity(f)$ :\\ $\alpha(z)(\pi\cdot i) = \sup L^z_{\pi\cdot i}$
\end{itemize}
By construction, $\alpha$ assigns any variable a well-formed, but
possibly infinite $\Sigma$-tree. We now have to show that $\alpha$ is
a solution of $\phi$, that is that $\alpha$ is a solution of any
atomic contraint contained in $\phi$:
\begin{enumerate}
\item $f(z_1,\ldots,z_n) \leq z \in \phi$

Let $\pi \in D(f(\alpha(z_1),\ldots,\alpha(z_n))) \cap
D(\alpha(z))$.

\begin{enumerate}

\item $\pi=\epsilon$:
We have that $f\in L^z_\pi$, hence by construction that
$\alpha(z)(\epsilon) \geq f$.

\item $\pi = i\pi'$:
We have that $L^{z_i}_{\pi'} \subseteq L^z_{\pi}$, hence
$\alpha(z)(\pi) = \sup L^z_{\pi} \geq \sup L^{z_i}_{\pi'} =
\alpha(z_i)(\pi') = f(\alpha(z_1),\ldots,\alpha(z_n))(\pi)$.

\end{enumerate}

\item $z \leq f(z_1,\ldots,z_n) \in \phi$

Let $\pi \in D(f(\alpha(z_1),\ldots,\alpha(z_n))) \cap
D(\alpha(z))$.
\begin{enumerate}

\item $\pi=\epsilon$:
Since $\phi$ is clash-free we have that for all $g \in L^z_\epsilon$
we have that $g\leq f$. Hence, $\alpha(z)(\epsilon) = \sup
L^z_\epsilon \leq f$.

\item $\pi = i\pi'$:
  
  Let $g\in L^z_\pi$. Then we know that $\phi \vdash i\pi'(g) \leq z$,
  that is there is an inequation $\leq h(\ldots,v_i,\ldots)\leq z$ with
  $\phi \vdash \pi'(g) \leq v_i$. By the last rule of the syntactic
  support of inequations we have that $\phi \vdash v_i \leq z_i$

\end{enumerate}

\end{enumerate}
\end{proof}

\begin{corr}
\label{lowersemilattice}
Let the signature $\Sigma$ be a lower semi-lattice with a greatest
element. Satisfiability of non-structural subtyping constraints in the
structure of possibly infinite trees can be checked in cubic time.
\end{corr}

\begin{proof}
By symmetry from Theorem~\ref{uppersemilatice}.
\end{proof}
